\chapter{Glosario de términos}

\textbf{Big Data}: Concepto que se refiere al almacenamiento de inmensas cantidades de datos.
\bigskip

\textbf{IOT}: Del inglés Internet of Things se refiere a la capacidad de dispositivos que poseen conexión a Internet y pueden comunicarse entre ellos así como con otros servicios
\bigskip

\textbf{Raspberry Pi}: Ordenador de bajo coste, con un procesaor ARM cuyo consumo es bajo.
\bigskip

\textbf{API Rest}: Arquitectura que trabaja sobre el protocolo HTTP y sirve para interactuar con bases de datos mediante ficheros como pueden ser JSON ó XML.
\bigskip

\textbf{Django Rest Framework}: Módulo de Python que permite crear APIs REST de una manera fácil e intuitiva.
\bigskip

\textbf{Django}: Framework en Python para crear páginas webs.
\bigskip


\textbf{Ajax}: Del inglés, Asynchronous JavaScript And XML provee métodos para el desarrollo de webs interactivas y sin recargas.
\bigskip


\textbf{Kaa}: Plataforma de IOT Open Source.
\bigskip


\textbf{SGBD}: Acrónimo de Sistema Gestor de Bases de Datos.
\bigskip


\textbf{Apache Cassandra}: SGBD orientado al almacenamiento masivo de información.
\bigskip


\textbf{Apache Zeppelin}: Visualizador web que soporta diversos formatos y lenguajes.
\bigskip


\textbf{Notebook}: Cuaderno de Zeppelin, en el se pueden escribir tareas para que sean ejecutadas y cuya salida es devuelta al usuario.
\bigskip


\textbf{ZigBee}: Conjunto de tecnologías que permiten la comunicación entre dispositivos de manera inalámbrica y con baja latencia.
\bigskip


\textbf{Open Source}: Dícese del software cuyo código es ofrecido públicamente para su estudio, modificación o simplemente uso.
\bigskip

\textbf{DAFO}: (FODA) es una análisis de debilidades, amenazas, fortalezas y oportunidades.
\bigskip

\textbf{Smart City}: concepto que se refiere al uso de tecnologías aplicadas sobre una ciudad.
\bigskip

\textbf{Apache Spark}: sistema de procesado de datos masivo que permite aplicar técnicas como machine learning sobre los mismos.
\bigskip

\textbf{Diagrama de PERT}: diagrama para organizar tareas y actividades por tiempo.
\bigskip

\textbf{Diagrama de GANTT}: al igual que el diagrama de Pert ofrece una visión del tiempo a emplear en las actividades programadas y como se encadenan unas con otras.
\bigskip

\textbf{RJ45}: conector de red común entre los equipos informáticos.
\bigskip

\textbf{NoSQL}: base de datos sin estructura relacional. Normalmente los datos son almacenados en formato de diccionario.
\bigskip

\textbf{Desarrollo Ágil}: Metodología que define la forma de trabajo sobre un proyecto en forma de iteraciones de manera que en cada una de éstas iteraciones se completa un requisito u objetivo del proyecto.
\bigskip

\textbf{Fork}: Abrir un nuevo camino dentro de un proyecto para orientar el desarrollo hacia otra dirección.
\bigskip


\textbf{Roadmap}: Camino a seguir para completar ciertos objetivos definidos.
\bigskip

\textbf{SO}: acrónimo de Sistema Operativo
\bigskip

\textbf{Python}: Lenguaje de programación cuya sintáxis se caracteriza por su flexibilidad y su legibilidad.
\bigskip

\textbf{LOPD}: acrónimo de Ley Orgánica de Protección de Datos.
\bigskip

\textbf{GitHub}: sistema de control de versiones que contribuye al desarrollo del software libre.
\bigskip

\textbf{Bootstrap}: librería que hace uso de jQuery y CSS para ofrecer interfaces amigables.
\bigskip

\textbf{Productor-Consumidor}: Paradigma dentro de la computación paralela que trata un problema mediante un elemento que genera datos mientras otro los procesa.
\bigskip

\textbf{Payload}: Información asociada a una petición.
\bigskip

\textbf{HTTP}: Del Inglés, HyperText Transfer Protocol que utilizan las direcciones de internet para interactuar con los servidores.
\bigskip

\textbf{VPS}: Del Inglés, Virtual Private Server. Servidor personal normalmente conectado a internet.
\bigskip

\textbf{JSON}: Del Inglés, JavaScript Object Notation, archivo con estructura de diccionario usado para el paso de datos de un sistema a otro.
\bigskip

\textbf{SQLite}: Motor de base de datos implementao en C cuyo formato es únicamente un archivo sobre el que se escribe la información.
\bigskip

\textbf{GET}: Dentro de HTTP, petición que demanda información.
\bigskip

\textbf{POST}: Dentro de HTTP, petición que envía información.
\bigskip

\textbf{PATCH}: Dentro de HTTP, petición que actualiza información.
\bigskip

\textbf{DELETE}: Dentro de HTTP, petición que elimina información.
\bigskip

\textbf{Backend}: Parte lógica de un software.
\bigskip

\textbf{Fronted}: Parte visual de un software.
\bigskip

\textbf{Middleware}: Pasarela intermedia por la que diversos elementos intercambian información.
\bigskip

\textbf{Webinars}: Pequeñas conferencias Online orientadas generalmente a la formación sobre un tema.
\bigskip


\textbf{CQL}: Acrónimo de Cassandra Language Query, es usado por Cassanra para interactuar con el motor de datos.
\bigskip

\textbf{CSS}: Del Inglés, Cascading Style Sheet, es un lenguaje estructurado y que da formato a elementos HTML.
\bigskip

\textbf{HTML5}: Lenguaje de marcas orientao a la creación de páginas webs estáticas.
\bigskip

\textbf{jQuery}: Biblioteca en JavaScript que provee funciones para interactuar con el código HTML de manera intuitiva.
\bigskip
