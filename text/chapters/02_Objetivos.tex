\chapter{Objetivos}

El principal objetivo del proyecto es ofrecer una aplicación web que muestre en tiempo real el estado del tráfico gracias a sensores \textit{(Raspberry Pi)} que analizan y envían la información obtenida de manera rápida y eficiente. Otro de los objetivos es ofrecer un almacenamiento robusto, escalable y replicable para manejar las grandes masas de información que se obtienen a lo largo de un día así como una herramienta para poder consultarlos y ofrecerlos de manera libre al público o entidades que deseen acceder a ellos.

\bigskip
Los objetivos se definen con más detalle a continuación:

\begin{itemize}
  \item \textbf{O1:} Desarrollar un método que permita reconocer cuando un vehículo ha pasado por la localización en la que el dispositivo recolector se encuentra.
  \item \textbf{O2:} Desarrollar una aplicación que permita a los usuarios instalar un dispositivo y aportar información del tráfico en el punto que ellos deseen.
  \item \textbf{O3:} Hacer uso de una base de datos especialmente diseñada para trabajar bajo el paradigma del Big Data.
  \item \textbf{O4:} Proveer un método fácil de consulta de los datos que el conjunto de dispositivos recolectores obtiene, para que cualquier usuario pueda consultarlos.
  \item \textbf{O5:} Dotar al sistema de una aplicación web en la que mediante un mapa, se muestren los puntos en los que están pasando vehículos.
\end{itemize}

\bigskip
Los aspectos formativos más utilizados son:
\begin{itemize}
  \item \textbf{Infraestructuras virtuales}, ya que los servicios usados son en su totalidad virtualizados así como el scripting necesario para la automatización de ciertas tareas.
  \item \textbf{Sistemas concurrentes}, debido a que al trabajar con dispositivos de un perfil medio-bajo para la recolección se hace necesario el paralelizar ciertas operaciones.
  \item \textbf{Bases de datos}, un elemento indispensable para tratar el problema que se plantea de almacenamiento de grandes cantidades de datos.
  \item \textbf{Ingeniería de servidores}, para la configuración de los distintos VPS usados.
  \item \textbf{Procesamiento digital de señales}, para un correcto entendimiento del análisis del sonido.
  \item \textbf{Desarrollo de aplicaciones para internet}, para el desarrollo de la plataforma web.
\end{itemize}

\section{Alcance de los objetivos}

El proyecto ofrecerá dos secciones en las que se podrá visualizar el estado del tráfico mediante un mapa y consultar los valores obtenidos de los dispositivos que se encuentran retransmitiendo información. Además permitirá a cualquier usuario configurar e iniciar su propio receptor para contribuir a la obtención de datos para el sistema.

\bigskip
En este caso, la aplicación hace uso del módulo desarrollado de detección de tráfico, sin embargo, el ámbito de la aplicación no se cierra exclusivamente a éste propósito ya que debido a su modularidad será posible el desarrollo y acoplamiento de distintos módulos de detección como podrían ser la temperatura, la humedad relativa o la luminosidad ó el análisis de los datos obtenidos mediante con herramientas como Apache Spark por ejemplo, quedando completamente abierto a cualquier modelo que un desarrollador quiera representar, por tanto hablamos de una herramienta completamente personalizable según las necesidades requeridas.


\section{Interdependencia de los objetivos}

Los objetivos podríamos alinearlos mentalmente a modo de una piramide inversa, el principal y sobre el que gira este proyecto es \textbf{O1} el cual va a generar la información, sin la cual el resto no tendrían sentido. Por otra parte existe una fuerte relación con \textbf{O2} ya que el mismo módulo encargado de generar la información deberá ser replicable y configurable tantas veces como se desee, de esta manera contribuimos a como se ha mencionado anteriormente, cualquier usuario pueda aportar datos al sistema.

En el siguiente escalón de la pirámide tendríamos \textbf{O3} y \textbf{O4} los cuales también están fuertemente ligados entre sí, ya que de nada nos serviría almacenar toda la información si no proveemos un método fácil para consultarla.

Por último tenemos \textbf{O5}, que al igual que los anteriores necesita consumir información, por tanto su dependencia con \textbf{O1}, es total.
