\chapter{Conclusiones y trabajos futuros}
\label{chap8}

\section{Conclusiones}
En un resumen general del estado de RSMap se puede determinar que está listo para poner en producción. Seguramente surgirían algunos incovenientes y esa es una de las \textit{competencias} que bajo mi punto de vista se adquieren cuando se estudia en la Universidad, más en concreto en mi experiencia siendo alumno de la \textbf{Escuela Técnica Superior de Ingenierías Informática y de Telecomunicación de la Universidad de Granada}. Me he demostrado la capacidad para resolver problemas, buscar información y alternativas ante los diferentes obstáculos que surgen cuando uno tiene que hacer combinar las diferentes piezas que conforman la aplicación.

El hecho de desarrollar un proyecto en el cual se usan muchos de los conceptos aprendidos durante el paso por las asignaturas hace que sea una experiencia enrriquecedora.

\bigskip

El proceso de análisis sin duda alguna ha sio uno de los más duros debido a la incertidumbre de si el camino elegido será la aproximación más acertada de cara al avance y finalización del proyecto.

Por otra parte, la elección de \LaTeX como soporte para la documentación ha acelerado por una parte el desarrollo de la misma generando muchos de los contenidos como índices, pies de foto o numeración sin embargo me ha supuesto un tiempo entender como funciona. No obstante ha sido una elección más que acertada en cuanto la calidad visual del documento.

\newpage

El proyecto se encuentra alojado bajo el dominio \url{www.rsmap.es} y todo el código y documentación se encuentran en \url{https://github.com/RSMap}. Si el lector desea ver la plataforma en funcionamiento, en el caso de que no le sea posible configurar su dispositivo de envío puede escribirme a \url{luqueburgosjm@gmail.com} y estaré encantado de activar mi dispositivo para que vea con más detalle el funcionamiento de RSMap.

\section{Trabajos futuros}

RSMap está abierto a muchas posibilidades que pasan desde anexar un sistema que estudie y saque conclusiones de los datos almacenados en Cassandra así como de monitorizar otro tipo de elementos ambientales gracias a que se pueden definir nuevas estructuras de almacenamiento y representación de una manera sencilla.

\bigskip

Haciendo un repaso por los objetivos planteados en un principio y los obtenios finalmente RSMap cuenta con las siguientes características:

\begin{itemize}
\item Permite reconocer cuano un vehículo pasa por un receptor si bien factores de ruio ambiental pueden influir de manera negativa sobre el resultado, sin embargo bajo condiciones óptimas las cuales son alcanzables en zonas no muy ruidosas o cuando existe menos ruido ajeno al tráfico los resultados son más que aceptables.
\item Tras una serie de pasos, cualquier usuario puede aportar información al sistema.
\item Los datos se encuentran almacenados en una base de datos con una potencia muy a tener en cuenta por tanto las técnicas de análisis de datos masivos son aplicables a RSMap.
\item La información que RSMap contiene es representada de una manera limpia y elegante.
\item Todos los componentes usados son Open Source.
\end{itemize}

Como reto personal me he propuesto seguir mejorando el algoritmo de detección de vehículos en ambientes con mucho ruido.
