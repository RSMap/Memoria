\chapter{Introducción}

A día de hoy, las tecnologías han llegado a un punto en el que envuelven completamente a las personas, ciudades y el mundo entero. Cada vez son más los elementos que nos encontramos a nuestro alrededor que poseen conexión a internet mediante la cual envían información de temperatura, humedad relativa, nuestras pulsaciones o los pasos que damos y que posteriormente podemos consultar en nuestro dispositivo móvil o el ordenador. Debido a esto y a otra gran diversidad de elementos que generan y almacenan cantidades ingentes de información han nacido nuevas ramas en la computación como son el \textbf{``Big Data''} o \textbf{``IOT''}, lo que se conoce comunmente como \textit{``el internet de las cosas.''}

\bigskip
Las ciudades sin duda alguna son uno de los mejores candidatos para el desarrollo y aplicación de éste tipo de tecnologías. Todos hemos oído hablar últimamente del término \textbf{``ciudad inteligente''} o \textbf{``smart city''} y es que, con la proliferación de las tecnologías citadas anteriormente así como una imperiosa necesidad de gestionar los recursos energéticos, económicos, medioambientales y administrativos de manera más eficiente han hecho necesaria la cohesión entre la tecnología y las propias ciudades.

\bigskip
Partiendo de esta base existen muchos campos en los que trabajar para hacer una ciudad mejor. Uno de los más relevantes puede ser el tráfico por diversos factores como pueden ser conocer su estado, crear sistemas reguladores del mismo así como poder comprobar los niveles de ruido generados en ciertas zonas.

\bigskip
Bajo estas premisas se aborda el problema de identificar el tráfico que fluye por una ciudad mediante el ruido que éste genera y poder almacenarlo con las tecnologías adecuadas para su análisis, el proceso técnico se encuentra detallado en el \textit{capítulo 5} (\textbf{Diseño}), que contiene el modelo que se pretende implementar y en el \textit{capítulo 6} (\textbf{Implementación}) que contiene los elementos software desarrollados.

Antes de ello se expondrán en el \textit{capítulo 2}  los (\textbf{Objetivos}) que se consideran necesarios para obtener una solución válida al problema; el \textit{capítulo 3} (\textbf{Planificación}) consta de las distintas fases a superar y en el \textit{capítulo 4}  (\textbf{Análisis}) de los requisitos necesarios a alcanzar.

\bigskip
Por último en el \textit{capítulo 7} (\textbf{Pruebas}), se remiten las pruebas realizadas que corroboran el correcto funcionamiento y se concluye con el el \textit{capítulo 8} (\textbf{Conclusiones y trabajos futuros}) en el cual se analiza el camino recorrido a lo largo de todas las fases anteriores así como posibles funcionalidades incorporables al sistema.
