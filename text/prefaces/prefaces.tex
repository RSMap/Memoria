\begin{center}
{\LARGE\bfseries\titulo}\\
\end{center}
\begin{center}
\autor\
\end{center}

\section*{Resumen}

\bigskip
\noindent{\textbf{Palabras clave}: \textit{smart cities}, \textit{tiempo real}, \textit{tráfico}, \textit{sensores}, \textit{análisis de datos}, \textit{software libre}\\

RSMap es un conjunto de tecnologías que trabajan en conjunto con el propósito de ofrecer información en tiempo real del estado del tráfico.

\bigskip
La información es recolectada por sensores situados cerca de cualquier ruta por la que pasen vehículos.

Éstos dispositivos filtran mediante micrófonos el sonido ambiente de la localización en la que se encuentran y remiten la información a los elementos que conforman la parte visual y de almacenamiento de RSMap.

\bigskip

En RSMap los dispositivos recolectores envían las señales de detección de vehículos a una plataforma web mediante una API Rest. Ésta web representará las señales enviadas sobre un mapa en tiempo real permitiendo a un usuario saber en qué puntos existe tráfico. La información almacenada aquí tiene caracter temporal pues, una vez representada sobre el mapa ya no es útil pero dado que se han invertido una serie de recursos en la recolección y procesado de dicha información, ésta es aprovechada para otros propósitos.

\bigskip
En este punto RSMap se sirve de herramientas orientadas al Big Data, guardando la información obtenida en una base de datos especialmente diseñada para almacenar millones de registros.

Además RSMap incluye otra forma de visualización, que permite consultar los datos almacenados de manera masiva a lo largo del tiempo para su análisis y estudio permitiendo así, la posibilidad de optimizar el tráfico de ciudades o carreteras, ver que rutas tienen más afluencia o hacer una estimación de el caudal de vehículos que circulará por una vía en un día.

\newpage
\begin{center}
{\LARGE\bfseries\tituloEng}\\
\end{center}
\begin{center}
\autor\
\end{center}

\section*{Extended abstract}

\bigskip
\noindent{\textbf{Key words}: \textit{smart cities}, \textit{real time}, \textit{traffic}, \textit{sensors}, \textit{data analysis}, \textit{open source}\\

The main reason to develop a system like RSMap is the innovative related technologies needed to cover a solution. Nowadays Big Data and IOT software and techniques are growing fast as well as the preoccupation for making the cities a better place to live improving several aspects like administrative, eco-friendly, and monetary subjects. The problem that RSMap tries to solve is the perfect scenario to join these topics and technologies in one solution.
Traffic is an important element which is related to the aspects mentioned above because with an optimal traffic in a city we may reduce pollution, identify noisy zones or get a better traffic flow.

\bigskip

RSMap system allows the end-user to know the actual traffic status through several devices which are capturing the noise generated  by vehicles and interpreting it as signals which are represented in two different ways.
The \textit{'obvious'} way is to show it on a map so we can see how many vehicles are running on certain street at a certain time.

The second way, much more interesting, is the possibility of watching the massive stored data along the days to perform Big Data analysis operations for example, to use it to configure predictive models.

\bigskip
The approach of the solution is acquired by 4 main elements.

The first of them are the receptors, They are lightweight computers, (Raspberry Pi in our case). Their purpose is the collector element in RSMap system, they process and send the signals obtained with a microphone to the storage/visualization points.

As middleware platform between receptors and storage RSMap uses KAA IOT which is a great open source platform that manages a lot of devices which task is to generate and simplify the way to allow the receptors to send data to the massive storage.

The third element is the massive storage, RSMap uses Apache Cassandra which is best-in-class database storage system for time series and its conditions of scalability makes it a perfect component to perform the tasks of storage being able to store hundreds of thousands registers per day.

Related to Apache Cassandra, RSMap makes use of Apache Zeppelin which is a very modern (but powerful) tool to visualize the data stored in Apache Cassandra in several ways due to the fact that we can get the data as we want through queries.

To conclude, RSMap has a typical end-point for normal users. A Web site where they can observe the traffic in real time, the receptors send signals when they 'think' that a vehicle passed in front of them.This Web site is developed in Django and Django-Rest-Framework. Django is a great web framework developed in Python to build professional web pages and Django-Rest-Framework is a module which allows the receptors to send detected signals to the web page with POST/PATCH/PUT/DELETE requests.


\bigskip
At the end, RSMap is a set of technologies and tools assembled to perform traffic detection. However, the possibilities to adapt the system to other purposes are too many thanks to the modular system structure. If we want to catch other kind of environment information we only have to adapt the receptors code and define new models to store the data but the essence of the application would be the same.

\newpage
\thispagestyle{empty}
\
\vspace{3cm}

\noindent\rule[-1ex]{\textwidth}{2pt}\\[4.5ex]

Yo, \textbf{\autor}, alumno de la titulación \textbf{\grado} de la \textbf{\escuela\ de la \universidad}, autorizo la ubicación de la siguiente copia de mi Trabajo Fin de Grado (\textit{\titulo}) en la biblioteca del centro para que pueda ser consultada por las personas que lo deseen.

\bigskip
Toda la información relacionada con el proyecto se encuentra bajo licencia \textbf{Creative Commons Attribution-ShareAlike 4.0} (\url{https://creativecommons.org/licenses/by-sa/4.0/}), por lo que se permite el uso comercial de la obra y de las posibles obras derivadas, la distribución de las cuales se debe hacer con una licencia igual a la que regula la obra original.

\vspace{4cm}

\noindent Fdo: \autor

\vspace{2cm}

\begin{flushright}
\ciudad, a \today
\end{flushright}

\newpage
\thispagestyle{empty}
\
\vspace{3cm}

\noindent\rule[-1ex]{\textwidth}{2pt}\\[4.5ex]

D. \textbf{\tutor}, profesor del \textbf{Departamento de Arquitectura y Tecnología de Computadores} de la \textbf{\universidad}.

\vspace{0.5cm}

\vspace{0.5cm}

\textbf{Informa:}

\vspace{0.5cm}

Que el presente trabajo, titulado \textit{\textbf{\titulo}}, ha sido realizado bajo su supervisión por \textbf{\autor}, y
autoriza la defensa de dicho trabajo ante el tribunal que corresponda.

\vspace{0.5cm}

Y para que conste, expide y firma el presente informe en \ciudad\ a \today.

\vspace{1cm}

\textbf{El tutor:}

\vspace{5cm}

\noindent \textbf{\tutor}

\chapter*{Agradecimientos}
\thispagestyle{empty}

\vspace{1cm}

A mis padres Manolo y Aurora, por haberme apoyado desde el primer al último día de mi paso por la Universidad y a mi hermana Ana María por ayudarme en lo que he necesitado.

\bigskip

A mi tutor Juan Julián Merelo Guervós por confiar en mí y permitirme realizar el proyecto bajo su supervisión.

\bigskip

A mis amigos porque siempre están ahí cuando los necesito.

\bigskip

A mis compañeros de Universidad por todos los buenos y no tan buenos momentos que hemos pasado juntos.

\bigskip

Y a todas las comunidades que hacen posible que el Software Libre sea un hecho, en las que se trabaja desinteresadamente por convicciónes propias y sin ánimo de lucro.
